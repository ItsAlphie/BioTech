% !TeX spellcheck = nl_NL
\documentclass[a4paper,kul]{kulakarticle} %options: kul or kulak (default)

\usepackage[utf8]{inputenc}
\usepackage[dutch]{babel}

\date{Academiejaar 2021 -- 2022}
\address{
	Industriële Ingenieurswetenschappen \\
	BioTechnologie \\
	Inge Holsbeeks \& Hans Rediers}
\title{Samenvatting}
\author{Robbe Decapmaker}
\usepackage{hyperref}
\usepackage{graphicx}
\usepackage{amsmath, amssymb, amsthm}
\usepackage{siunitx}
\usepackage{flafter} 
\usepackage{pdfpages}



\begin{document}

\maketitle

\section*{Inleiding}

De samenvatting van BioTechnologie. \href{https://github.com/debber1/BioTech}{De source code is te vinden op Github.}\\
%DEZE ZIN IS ENKEL RELEVANT TIJDENS DE ONTWIKKELING VAN DIT DOCUMENT
\textbf{Dit document is een `work in progress', dit wil zeggen dat er (ongeveer) een wekelijkse update zal zijn. De meest recente versie zal altijd op Github staan!}
\tableofcontents
\section{Koolhydraten}
\subsection{Monosachariden}
\subsubsection{Naamgeving}
\subsubsection{Voorstellingen}
\subsubsection{Stereochemie}
\subsubsection{Belangrijke monosachariden}
\subsubsection{Afgeleiden}
\subsubsection{Reducerende koolhydraten}
\subsection{Disachariden}
\subsubsection{Belangrijke disachariden}
\subsection{Polysachariden}
\subsubsection{Belangrijke polysachariden}
\section{Lipiden}
%Overzicht Schema slide 28
\subsection{Biologische functies van lipiden}
\subsection{Vetzuren}
\subsubsection{Structuur}
\subsubsection{(On)Verzadigde vetzuren}
\subsubsection{Cis- en Transvetzuren}
\subsubsection{Omega vetzuren}
\subsubsection{Reacties met vetzuren}
%Mooie samenvatting op slide 16
\subsection{Glyceriden}
\subsubsection{Structuur}
\subsubsection{Triglyceriden}
\subsubsection{Reacties}
\subsubsection{Fosfoglyceriden}
\subsection{Niet-glyceride lipiden}
\subsubsection{Sfingolipide}
\subsubsection{Steroïden}




\end{document}
